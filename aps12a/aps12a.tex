\documentclass[10pt]{article}
\usepackage{graphicx}

\begin{document}

\title{Modellazione di dominio (prima parte)}
\author{drew}

\section{Introduzione}
Un modello di dominio è un modello concettuale a oggetti che descrive le informazioni che devono essere gestite dal sistema.\\
Viene considerato l'elaborato più importante della OOA ed è sviluppato in modo iterativo ed incrementale.
\subsection{Che cos'è un modello di dominio}
Un modello di dominio è la rappresentazione visuale di classi concettuali o di oggetti del mondo reale, nel dominio di interesse, e delle relazioni tra di essi. Viene anche chiamato \textbf{modello concettuale} o \textbf{modello degli oggetti di dominio}.

\section{Come rappresentare un modello di dominio}
Un modello di dominio può essere realizzato in UML da un punto di vista concettuale, come diagramma delle classi, ovvero classi concettuali, associazioni e attributi.
\subsection{Classi, associazioni, attributi}
Una \textbf{classe} rappresenta una cosa o un concetto (nel dominio di interesse), ovvero un insieme di cose o concetti (oggetti) con caratteristiche ritenute simili.\\
Un'\textbf{associazione} rappresenta una relazione (un insieme di collegamenti) tra gli oggetti di due classi.\\
Un attributo rappresenta una proprietà elementare (un dato) degli oggetti di una classe.
\subsection{Esempio (di porzione) di modello di dominio}
\begin{center}
\includegraphics[scale=0.5]{./images/porzionemodellodominio.jpg}
\end{center}
Importante notare come non un modello di dominio non mostra componenti software nè responsabilità. 
\paragraph{} Esistono due diversi utilizzi del termine "modello di dominio" nella pratica.\\
In UP (Unified Process) ed in questo corso il \textbf{modello di dominio} è la descrizione concettuale di oggetti del mondo di reale interesse.\\
Nel software, un modello di dominio è lo strato software composto dalle classi software usate per rappresentare oggetti del mondo reale, anche chiamato \textbf{strato di dominio}.\\
Il modello di dominio e lo strato di dominio danno ciascuno forma all'altro, ed è quindi questa relazione profonda tra il modello di dominio e la sua implementazione che lo rende importante.
\section{Perchè creare un modello di dominio}
\begin{itemize}
\item nell'analisi
\begin{itemize}
\item per comprendere il dominio del sistema da realizzare e il suo vocabolario
\item per definire un linguaggio comune che abiliti la comunicazione tra le varie parti interessate
\end{itemize}
\item per la progettazione
\begin{itemize}
\item come fonte di ispirazione per progettare lo strato del dominio
\item per mantenere basso il salto rappresentazionale
\item per abilitare lo sviluppo di un sistema software facilmente comprensibile, modificabile ed evolvibile
\end{itemize}
\end{itemize}



\paragraph{•}


\end{document}