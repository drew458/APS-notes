\documentclass[10pt]{article}

\begin{document}

\title{Modellazione di dominio (prima parte)}
\author{drew}

\section{Introduzione}
Un modello di dominio è un modello concettuale a oggetti che descrive le informazioni che devono essere gestite dal sistema.\\
Viene considerato l'elaborato più importante della OOA ed è sviluppato in modo iterativo ed incrementale.
\subsection{Che cos'è un modello di dominio}
Un modello di dominio è la rappresentazione visuale di classi concettuali o di oggetti del mondo reale, nel dominio di interesse, e delle relazioni tra di essi. Viene anche chiamato \textbf{modello concettuale} o \textbf{modello degli oggetti di dominio}.

\section{Come rappresentare un modello di dominio}
Un modello di dominio può essere realizzato in UML da un punto di vista concettuale, come diagramma delle classi, ovvero classi concettuali, associazioni e attributi.
\subsection{Classi, associazioni, attributi}
Una \textbf{classe} rappresenta una cosa o un concetto (nel dominio di interesse), ovvero un insieme di cose o concetti (oggetti) con caratteristiche ritenute simili.\\
Un'\textbf{associazione} rappresenta una relazione (un insieme di collegamenti) tra gli oggetti di due classi.\\
Un attributo rappresenta una proprietà elementare (un dato) degli oggetti di una classe.
\subsection{Esempio di (porzione) di modello di dominio}
\includegraphics[scale=0.5]{./images/porzionemodellodominio.jpg}

\paragraph{•}


\end{document}