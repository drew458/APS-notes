\documentclass[11pt]{article}




\begin{document}
\title{Casi d'uso}
\author{drew}
\maketitle
\section{Introduzione}
I casi d'uso sono storie scritte che descrivono il funzionamento del sistema che deve essere realizzato. Sono utili per identificare e registrare requisiti e possono fare da guida per i progetti di sviluppo (come l'OOA/D).\\
\paragraph{Esempio}
\textbf{Elabora Vendita:}
Un cliente arriva alla cassa con alcuni articoli da acquistare. Il 
cassiere utilizza il sistema POS per registrare ogni articolo 
acquistato. Il sistema mostra il totale e i dettagli per ogni articolo. Il 
cassiere inserisce informazioni sul pagamento, che il sistema 
convalida e registra. Il sistema aggiorna l'inventario. Il cliente 
riceve dal sistema una ricevuta e poi se ne va con gli articoli
acquistati.
\section{Attori, scenari, casi d'uso}
Un \textbf{attore} è qualcosa o qualcuno con un certo comportamento  che interagisce con il sistema.\\
Uno \textbf{scenario} è una sequenza specifica di interazioni tra il sistema e gli attori.\\
Un \textbf{caso d'uso} è una collezione di scenari correlati, di successo e 
fallimento, che descrivono un attore che usa un sistema per 
raggiungere un obiettivo.\\
\section{Vantaggi dei casi d'uso}
I casi d'uso hanno molti vantaggi, tra cui l'essere uno strumento semplice comprensibile dagli utenti, una base per i test di sistema e l'essere comunque basati su scenari reali. \\
L'unico limite è costituito dal fatto che non sono sempre applicabili.\\
I casi d'uso sono requisiti (soprattutto funzionali) e hanno uno stile più moderno rispetto ad una più antiquata elencazione di funzioni e caratteristiche funzionali.
\section{Tipi di Attori}
Esistono diversi tipi di attori:
\begin{itemize}
	\item \textbf{Attore primario}\\
		Usa il SuD (Sistema in Discussione) affinché vengano raggiunti degli 			obiettivi. 
	\item \textbf{Attore finale}\\
		Vuole raggiungere degli obiettivi
	\item \textbf{Attore di supporto}\\
		Offre dei servizi al SuD
	\item \textbf{Attori fuori scena}\\
		Ha interessi nel comportamento del SuD
\end{itemize}
\section{Formati comuni per i casi d'uso}
I casi d'uso possono essere scritti in vari formati, con un grado 
variabile di formalità e dettaglio.\\
\begin{itemize}
	\item formato \textbf{breve}
	\item formato \textbf{informale}
	\item formato \textbf{dettagliato}
\end{itemize}
\section{Sezioni di un caso d'uso dettagliato}
Il preambolo contiene informazioni (opzionali) che è importante leggere prima dello scenario principale.
\subsubsection{Portata}
Confini del SuD.
\subsubsection{Livello}
Spesso livello di obiettivo utente.
\subsubsection{Attore finale e attore primario}
\begin{itemize}
	\item attore finale - chi vuole raggiungere un obiettivo dall'uso del sistema
	\item attore primario - chi vede direttamente i servizi al sistema per raggiungere quell'obiettivo
\end{itemize} 
\subsubsection{Parti interessate e interessi}
Molto importante, perché suggerisce e limita ciò che deve fare il sistema.\\
\begin{itemize}
	\item "Il sistema concretizza un contratto tra le parti interessate, con 
		i casi d'uso che descrivono gli aspetti comportamentali del 
		contratto".
\end{itemize}
Costituisce uno strumento accurato per identificare i requisiti e valutare la loro utilità.
\subsubsection{Pre-condizioni}
Che cosa deve essere sempre vero prima di iniziare uno scenario
del caso d'uso.
\subsubsection{Garanzia di successo (post-condizioni)}
Che cosa deve essere vero quando il caso d'uso (qualsiasi scenario) viene completato con successo.
Deve soddisfare gli interessi di tutte le parti interessate.
\subsubsection{Garanzia minima}
Che cosa deve essere vero quando il caso d'uso viene completato (successo o fallimento).
Deve proteggere gli interessi di tutte le parti interessate.
\subsubsection{Scenario principale di successo (flusso di base)}
Costituisce il percorso tipico di successo che soddisfa gli interessi 
delle parti interessate – "percorso felice" o "happy path".
Formato da una sequenza di passi, che possono essere ripetuti, e non comprende solitamente salti condizionali.
\section{Estensioni (flussi alternativi)}
Sono molto importanti, perché descrivono tutti i tipi di scenari (di successo o di fallimento).\\
Costituiscono la maggior parte del testo.\\
Le estensioni sono descritte come deviazioni (condizionali) dallo scenario che le contiene.\\
\\
Ogni estensione è formata da due parti:
\begin{itemize}
	\item \textbf{Condizione} - Qualcosa che può essere rilevato da un sistema o da un attore
	\item \textbf{Gestione}
		Una sequenza di uno o più passi
\end{itemize}
Dopo l'esecuzione di un'estensione, il controllo torna di solito allo 
scenario principale (o allo scenario che contiene l'estensione). Sono possibili anche estensioni di estensioni.\\
Esistono tre tipi di estensioni.
\subsubsection{Primo tipo}
L'attore primario vuole far procedere il caso d'uso diversamente da quanto previsto nello scenario principale di successo.\\
In questo caso l'attore interagisce in modo diverso da quanto previsto nello scenario principale.
\subsubsection{Secondo tipo}
Si verifica qualcosa (ad es., un errore) per cui il caso d'uso deve procedere diversamente da quanto previsto nello scenario principale di successo.\\
In questo caso di solito è il sistema che se ne accorge mentre esegue un'azione o effettua una validazione.
\subsubsection{Terzo tipo}
Un passo dello scenario principale descrive un'azione "astratta" – il caso d'uso contiene poi un'estensione di quel passo per ciascun modo "concreto" di eseguire quell'azione.

\section{Scrivere in uno stile essenziale}
\subsubsection{Linea guida}
Nell'analisi dei requisiti, scrivi i casi d'uso in uno \textbf{stile essenziale}.\\
Ignora l'interfaccia utente.\\
Fai riferimento alle intenzioni degli utenti e alle responsabilità del sistema – non ad azioni concrete.
\subsubsection{Stile essenziale}
L'amministratore si identifica.\\
Il sistema autentica l'identità.
\subsubsection{Stile concreto}
L'amministratore immette ID e password in una finestra di dialogo.\\
Il sistema autentica l'amministratore.\\
Il sistema mostra la finestra "edit users".

\section{Scrivere in modo coinciso ma chiaro}
\subsubsection{Linea guida}
Scrivi i casi d'uso in modo conciso – ma allo stesso tempo chiaro e completo.\\
Scrivi chiaramente ed esplicitamente soggetto, verbo ed eventuali frasi subordinate – verbo al tempo presente e nella forma attiva.
\subsubsection{Ad esempio}
Il sistema autentica l'amministratore.
\subsubsection{E non}
L'amministratore viene autenticato.

\section{Casi d'uso a scatola nera}
I casi d'uso più comuni sono quelli a scatola nera.\\
Specificano cosa deve fare il sistema, senza descrivere come lo fa.\\
Il sistema ha delle responsabilità, e collabora con gli attori, elementi che hanno altre responsabilità.\\
Non descrivono la struttura interna del sistema, né il suo progetto.
\subsubsection{Stile a scatola nera}
Il sistema registra la vendita.
\subsubsection{E non}
Il sistema memorizza la vendita in una base di dati.\\
Il sistema genera un comando SQL di tipo INSERT per memorizzare la vendita.

\section{Livello dei casi d'uso}
I casi d'uso possono essere scritti a livelli diversi , quindi per ciascun caso d'uso è importante capire qual è il suo livello.
\subsection{Livello di obiettivo utente}
Il più interessante nell'analisi dei requisiti - corrisponde ai casi d'uso EBP
\subsection{Livello di sotto-funzione}
Utili per mettere a fattore comune parti di casi d'uso e/o per descrivere interazioni di dettaglio.
\subsection{Livello di sommario}
Un caso d'uso a questo livello comprende più casi d'uso a livello di obiettivo utente – la sua durata può essere ore,  giorni, mesi o anche anni – utile per l'identificazione dei casi d'uso EPB, e per comprendere/descrivere il loro contesto.

\section{Diagrammi dei casi d'uso}
I diagrammi dei casi d'uso di UML sono una notazione grafica per mostrare i nomi dei casi d'uso e degli attori, e le loro relazioni.\\
Sono molto meno espressivi dei casi d'uso, ma sono utili per complementare la lista attori-obiettivi e descrivere il contesto di un sistema e i suoi confini.\\
I diagrammi dei casi d'uso sono secondari nella lavorazione dei casi d'uso.
\subsection{Linee guida}
\begin{itemize}
	\item  disegna un semplice diagramma 			dei casi d'uso insieme a una lista 			attori-obiettivi.
	\item  attento a non passare troppo 			tempo a disegnare diagrammi dei 			casi d'uso 
\end{itemize}

\section{Casi d'uso nei processi iterativi}
I casi d'uso sono centrali in UP e in altri processi iterativi.\\
UP incoraggia lo sviluppo guidato dai casi d'uso, in cui vengono registrati i requisiti.\\
I casi d'uso hanno un ruolo importante della pianificazione iterativa.\\
La progettazione è guidata dalla realizzazione dei casi d'uso.\\
I casi d'uso influenzano l'organizzazione dei manuali d'uso .\\
I test funzionali e di sistema possono essere basati su scenari di casi d'uso.

\end{document}