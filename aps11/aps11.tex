\documentclass[10pt]{article}

\begin{document}

\title{Verso l'analisi a oggetti}
\author{drew}
\maketitle

\section{Introduzione}
L'analisi enfatizza un'investigazione di un problema e dei suoi requisiti, quindi è interessata al che cosa, non al come e non è interessata direttamente alle soluzioni del problema.\\
Esistono diversi "metodi" per effettuare l'analisi, ciascuno dei quali è in qualche modo legato a una "strategia risolutiva" che si vuole adottare o a uno specifico "tipo di sistema" che si vuole realizzare.\\
\textbf{L'analisi orientata agli oggetti (OOA)} si basa principalmente sull'identificazione dei concetti nel dominio del problema – e su una loro descrizione a oggetti.
L'analisi del software ha lo scopo di comprendere il problema che deve essere affrontato dal sistema software di interesse – tre aspetti significativi:
\begin{itemize}
\item le informazioni che il sistema deve gestire (dominio informativo)\\
Nel metodo di analisi orientata agli oggetti vengono descritte dal \textit{modello di dominio}.
\item le funzioni (o operazioni) che il sistema dovrà svolgere 
Nel metodo di analisi orientata agli oggetti vengono descritte dalle \textit{operazioni di sistema} e \textit{diagrammi di sequenza di sistema}.
\item il comportamento del sistema
Nel metodo di analisi orientata agli oggetti viene descritto dai \textit{contratti} delle operazioni di sistema.
\end{itemize}

\end{document}