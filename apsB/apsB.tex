\documentclass[10pt]{article}

\usepackage{graphicx}
\graphicspath{ {./images/} }

\begin{document}

\title{Dalla progettazione concettuale alla modellazione di dominio}
\author{drew}
\maketitle

\textbf{Disclaimer:} la parte di modellazioni di dominio, in un compito d'esame standard, vale circa 9/30. Quindi \textbf{ATTENZIONE!!!}

\section{Introduzione}
Nell’OOA, la modellazione di dominio ha lo scopo di descrivere le informazioni della realtà di interesse secondo una rappresentazione concettuale e a oggetti.\\
In questo contesto, la “realtà di interesse” viene chiamata “dominio del problema” . Dunque, progettazione concettuale e modellazione di dominio 
hanno scopi simili. Vedremo anche che sono attività svolte con strumenti e metodi simili.

\section{Terminologia}
Rispetto alla progettazione concettuale di Basi di Dati, la quale è molto simile alla modellazione di dominio, la terminologia è un po' diversa.
\begin{itemize}
\item i diagrammi si chiamano \textbf{modelli}
\item i formalismi per esprimere i modelli si chiamano \textbf{linguaggi} - ad esempio il linguaggio UML
\end{itemize}
Il risultato della modellazione di dominio è un \textbf{modello di dominio}, espresso mediante un \textbf{diagramma delle classi di UML}, usato dal punto di vista concettuale.
\subsection{Entità e classi concettuali}
Una \textbf{classe concettuale} rappresenta un insieme di cose o concetti con caratteristiche simili.
\begin{center}
\includegraphics[scale=0.5]{classe_concettuale} \\
\end{center}
Le relative istanze sono chiamate \textbf{istanze} o \textbf{oggetti}.
\subsection{Attributi}
Un \textbf{attributo} rappresenta una proprietà elementare degli oggetti di una classe.
\begin{center}
\includegraphics[scale=0.5]{attributi}
\end{center}
\subsection{Relazioni e Associazioni}
Un'\textbf{associazione} rappresenta una relazione (una connessione significativa) tra classi
\begin{center}
\includegraphics[scale=0.5]{relazioni_associazioni} \\
\end{center}
Le istanza di un'associazione si chiamano collegamenti.\\
Le associazioni sono in genere binarie - le N-arie sono possibili, ma poco comuni.\\
Si possono rappresentare associazioni con attributi, ma è poco comune.\\
Il nome è generalmente un verbo che indica una relazione.
\subsection{Cardinalità e Molteplicità}
Una molteplicità indica quante istanze di una classe possono essere associate a un'istanza dell'altra classe.
\begin{center}
\includegraphics[scale=0.5]{cardinalità_molteplicità}
\end{center}
\subsection{Altre differenze significative}
Differenze significative tra ER (Entity-Relationship) e UML.\\
UML:
\begin{itemize}
\item per rappresentare tutte le informazioni che devono essere gestite da un'applicazione
\item  una classe rappresenta una classe di oggetti – ma non il 
relativo "insieme"
\item può essere ok avere classi che hanno un solo oggetto
\item  può essere ok avere classi senza attributi – con cautela
\item può essere ok avere classi che rappresentano solo comportamento – con cautela 
\end{itemize}

\section{Dalla PCBD alla MD}
Alcune idee importanti della modellazione di dominio corrispondono (con alcune varianti) a quelle della progettazione concettuale di basi di dati
\begin{itemize}
\item  bisogna rappresentare le informazioni del dominio del problema – senza preoccuparsi dell'implementazione di queste 
informazioni 
\item  bisogna concentrarsi solo su ciò che è rilevante ai fini della descrizione del dominio del problema
\item  le classi concettuali rappresentano classi di oggetti o fatti che hanno proprietà comuni ed esistenza autonoma
\item   le associazioni rappresentano legami logici significativi tra due classi concettuali 
\item gli attributi descrivono le proprietà elementari delle classi 
concettuali
\end{itemize}
\subsection{Criteri generali di rappresentazione}
I criteri generali di rappresentazione sono:
\begin{itemize}
\item   se un concetto ha proprietà significative e/o descrive una classe (insieme) di oggetti con esistenza autonoma, è opportuno 
rappresentarlo come una classe concettuale 
\item  se un concetto ha una struttura semplice e pensiamo ad esso 
come ad un valore, è opportuno rappresentarlo come un attributo, associato alla classe concettuale a cui si riferisce
\item  se sono state individuate due classi concettuali e c’è un concetto che le associa, questo concetto può essere rappresentato come un'associazione
\end{itemize}
Le \textbf{strategie} sono:
\begin{itemize}
\item top down
\item bottom up
\item inside out (a macchia d'olio)
\item mista
\end{itemize}

\section{Esempio parziale modello di dominio}
\begin{center}
\includegraphics[scale=0.5]{esempio_modello_dominio}
\end{center}

\section{Oggetti di dominio}
Un diagramma degli oggetti di dominio rappresenta, in modo visuale, un insieme di oggetti di esempio del mondo reale – nel contesto del dominio del problema.\\
Formato da un grafo di oggetti, dove ogni nodo rappresenta un oggetto - etichettato con il nome di una classe.\\
Ciascun oggetto è decorato con valori per gli attributi della classe.\\
Ciascun arco rappresenta un collegamento - etichettato con il nome di un'associazione (tra quelle che collegano le classi di due oggetti).\\
\\
Oggetti di dominio per rappresentare uno studente Mario Rossi, nato 
a Roma, che frequenta i corsi di SIW e APS:
\begin{center}
\includegraphics[scale=0.8]{esempio_oggetti_dominio}\\
\end{center}
Oggetti di dominio per rappresentare:
\begin{itemize}
\item  uno studente Mario Rossi, nato a Roma, frequenta SIW e APS
\item  una studentessa Eva Verdi, nata a Roma, frequenta APS
\end{itemize}
\begin{center}
\includegraphics[scale=0.8]{esempio_oggetti_dominio2}\\
\end{center}








\end{document}