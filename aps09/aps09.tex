\documentclass[10pt]{article}

\begin{document}

\title{Storie Utente}
\author{drew}
\maketitle

\section{Introduzione}
Le storie utente sono un formato per i requisiti usato nei metodi agili.\\
Sostengono una collaborazione stretta e continuativa con il cliente, basata su conversazione frequenti - anziché di scrivere i requisiti in modo dettagliato e formale.\\
\\
Una storia descrive, in modo semplice, una funzionalità o un requisito che è di interesse per un utente del sistema.\\
Le storie vengono solitamente scritte su piccole schede. Sul retro della scheda, invece, vengono riportati i test di accettazione, ovvero le aspettative degli utenti.\\
\section{Perché le storie utente}
Le motivazioni alla base delle storie utente sono:
\begin{itemize}
\item Il coinvolgimento continuo degli utenti nei progetti software è molto importante
\item La comunicazione faccia a faccia può essere preferibile alla scrittura di documentazione
\item Sono compatibili con lo sviluppo iterativo e agile
\end{itemize}


\end{document}